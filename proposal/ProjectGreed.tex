\documentclass{report}
\usepackage[a4paper]{geometry}
\usepackage{tabularx}
\usepackage{amsmath}

\usepackage{biblatex}
\addbibresource{sources.bib}

\usepackage[OT1]{fontenc}
\renewcommand*\familydefault{\sfdefault}

% \fontfamily{cmss}\selectfont

\title{\huge Project Proposal: Evaluating the effectiveness of different trading algorithms}
\author{
BL (z5592091), TRJ (z5592097),\\
KM (z5592102), VP (z5592103), YM (z5624648) 
}
\date{}

\begin{document}
\maketitle

\section*{Executive summary}

Algorithms are used extensively in the stock market in order to optimise trade timings,
decision, and portfoios to maximise profit and reduce risk.
These algorithms rely on large amounts of historical market data to identify patterns and finely tune parameters 
in order to generate and execute decisions, removing the reliance on human intuition and tranforming
the problem into an optimisation and single-processing one.
Furthermore, algorithmic trading allows trades to be executed at much higher speed than possible for humans,
and also allows for objective testing of performance, bringing more avenues for improvement.

\subsection*{Context and Background Information}

As Treleaven, Galas \& Lalchand (2013) explain, trading algorithms and systems aim to remove human emotional bias
and increase efficiency by processing large volumes of market data in real time.
The field has evolved from simple rule-based systems to highly complex models involving statistical and machine learning methods,
and the added precision allows for high-frequency trading not possible before.
Despite this evolution, there is still a lot of value to be had in empirically analysing simple technical-rule strategies
due to their transparency and interpretability, as they can highlight the need for complexity in real-world algorithms.

Algorithmic trading systems primarily aim to maximise profit, but also manage risk and minimise transaction costs
of various financial securities, including stocks, bonds, options, and more.
A large part of the effort in algorithmic trading comes from finding and parsing quality data,
which is important to test and run many algorithms.
In a full algorithmic trading system, models for calculating and taking into account transaction costs
and the specifics of trade execution need to be implemented as well. \cite{treleaven_algorithmic_2013}

\section*{Survey of existing trading algorithms}

In his research paper, Hägg (2023) identifies different types of trading algorithms. \cite{hagg_study_2023}

\subsection*{Systematic}

This type of trading executes a defined set of rules consistently across trades and specifies
entry and exit strategies.
The core aim is repeatable, consistent execution of a fixed strategy.

\subsection*{Quantative}

Quantitative (black-box) trading uses secret, model-driven rules to systematically extract
information and optimise porfolios rooted in Markowitz's \emph{Modern Portfolio Theory},
which balances risk and return by solving for efficient asset weights combining risky and risk-free assets.

\subsection*{Statistical Arbitrage}

Statistical Arbitrage is a systematic trading approach that utilizes real-time and historical data analysis
to take advantage of mispricings while minimising overall risk.
Innovative tools from science and economics are frequently used in statistical arbitrage, including time series,
data mining, artificial intelligence, agent-based models, and fractals.
Finding opportunities to take advantage of market inefficiencies while managing risk is the aim of statistical arbitrage.

For our project, we will be testing different Systematic Algorithms as they require the creation of algorithms/strategies that maximises our portfolio, some algorithms we found are popularly used in the industry today include:
\begin{itemize}
  \item \textbf{Trend Followers}:
    A class of algorithms that perform statistical analysis on the trend to predict the next movement,
    and make trade decisions based on this data

  \item \textbf{Pair Trading}:
    Tracking two correlated assets (for example, Coke and Pepsi).
    If their spread deviates beyond some threshold,
    short the overpriced asset and long the underpriced one.
    This requirements multiple assets and introduces cointegration testing.

  \item \textbf{Moving Average Crossover}:
    This compares short-term and long-term moving average (MA) signals.
    Some variants of these are exponential MAs, triple crossover, and weighted MAs.

  \item \textbf{Reinforcement Learning (Deep Q / Policy Gradient)}
    Instead of tabular Q-learning, this uses a small neural network to approximate Q-values or policies.
    stuff

  \item \textbf{Kelly Criterion Position Sizing}:
    Not a signal strategy, but a risk management overlay.
    This dynamically sizes trades based on experience return or variance.
    It can be applied on top of other strategies.

\end{itemize}

\subsection*{Testing trading algorithms}

The most common method of testing trading strategies and algorithms in through backtesting, that is,
checking their effectiveness on historical data.
The simplest way of backtesting is fetching historical prices and simulating the algorithm.
An event driven backtesting system (one where the algorithm runs in a loop and executes actions
when certain events occur) can help create a realistic simulation that avoids anticipation bias,
and can be made portable to load different data sets to thoroughly test the performance of each algorithm.
We can use this to test with volatile and stable assets.
There are variations and more advanced method ofs backtesting, including generating simulated data sets
using generative models that are fed historical data to create testing conditions.
\cite{koshiyama_generative_2019}
\cite{lezmi_improving_2020}

Known drawbacks to backtesting include overfitted algorithms, which are when they are more suited to
some data sets than others, and unreliable performance during black swan (e.g. COVID-19) events.
\\\ \\
Another method of testing algorithms is through forward testing (walk forward optimisation),
which utilises live data to test algorithms.
A basic way of doing this is to take a live time interval, use 70\% of the earlier (in sample) data
to make decisions and/or predictions about the remaining 30\% (out sample) data.
This works on live or near-live data and tests an algorithm's effectiveness.

The benefits of this method include validating algorithmic execution in real time to build
trust in the algorithm, and being able to factor in latency.
But this only tests an algorithm on its performance in a specific market, which can fluctuate.
So it must be tested in various conditions.

\\\ \\
Once algorithms have been tested, they must be evaluated via metrics and benchmarks.
These metrics have to measure the important outcomes of the algorithm, such as profit and risk,
which metrics like profit factor and maximum drawdown measure respectively.
Metrics like the Sharpe Ratio combine the outcomes, measuring risk-adjusted returns.
\cite{tradetron_how_2023}

\section*{Research plan}

\subsection*{Project Timeline}

\renewcommand{\arraystretch}{1.7}
\begin{center}
\begin{tabular}{m{0.2\textwidth}m{0.35\textwidth}m{0.3\textwidth}}
  \hline
  \textbf{Date} & \textbf{Aim} & \textbf{Notes}\\ \hline\hline
  Sunday, 5 Oct & Finish project proposal & \\
  Tuesday, 7 Oct & Create function to test algorithms & Coded in a real language (likely Python) \\
  Friday, 10 Oct & Find data for experiments & \\
  Sunday, 12 Oct & First algorithm & \\
  Sunday, 19 Oct & Second/Third algorithms & Split into two teams, each working on one algorithm \\
  \textit{Flex week} & Fourth algorithm & \\
  Tuesday, 28 Oct & Fifth algorithm & Based on previous algorithms and research \\
  Sunday, 2 Nov & Progression check writeup & \\
  Monday, 3 Nov & Start working on report and poster & If time permits, we may start work on a $6^{\textsf{th}}$ algorithm \\
  Sunday, 9 Nov & Soft deadline for poster and report & \\
  Wed/Fri, Week 10 & Poster showcase & \\
  \hline
\end{tabular}
\end{center}

When planning algorithms, we will write them in plain English as well as implementing them in
a language to document and test them.

\printbibliography

\end{document}
